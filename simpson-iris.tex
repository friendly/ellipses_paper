\subsection{Simpson's paradox, marginal and conditional relationships}\label{sec:simpson-iris}

Because it provides a visual summary of means, variances, and correlations,
the data ellipse is ideally suited as a tool for illustrating and
explicating various
phenomena that occur in the analysis of linear models.
One class of simple, but important, examples concerns the difference between the marginal
relationship between variables, ignoring some important factor or covariate,
and the conditional relationship, adjusting (controlling) for that
factor or covariate.

\begin{figure}[htb]
  \centering
  \includegraphics[width=\textwidth,clip]{fig/contiris3}
  \caption{Marginal (a), conditional (b), and pooled within-sample (c) relationships
  of Sepal length and Sepal width in the iris data. Total-sample data ellipses are
  shown as black, solid curves; individual-group data and ellipses are shown with
  colors and dashed lines}%
  \label{fig:contiris3}
\end{figure}

Simpson's paradox \citep{Simpson:51} occurs when the marginal and
conditional relationships differ in direction. This may be seen in the plots
of Sepal length against Sepal width for the iris data shown in \figref{fig:contiris3}. Ignoring
iris species, the marginal, total-sample correlation is slightly negative
as seen in panel (a). The individual-sample ellipses in panel (b) show
that the conditional, within-species correlations are all positive, with
approximately equal regression slopes.  The group means have a negative
relationship, accounting for the negative marginal correlation.

A correct analysis of the (conditional) relationship between these variables, controlling or adjusting for mean
differences among species, is based on the pooled within-sample covariance matrix,
  \begin{equation} \label{eq:Sp}
  \mat{S}_{\textrm{within}}  = (N - g)^{-1}
  \sum_{i=1}^g
  \sum_{j=1}^{n_i}
  ( \vec{y}_{ij}  -  \bar{\vec{y}}_{i.} )
  ( \vec{y}_{ij}  -  \bar{\vec{y}}_{i.} ) \trans
  =(N - g)^{-1}
  \sum_{i=1}^g
  (n_i - 1) \mat{S}_i
  \comma
  \end{equation}
where $N = \sum n_i$, and the result
is shown in
panel (c) of \figref{fig:contiris3}.
In this graph, the data for \emph{each} species were first
transformed to deviations from the species means on both variables
and then translated back to the grand means.

In a more general context, $ \mat{S}_{\textrm{within}}$
appears as the $\mat{E}$ matrix in a multivariate
linear model, adjusting or controlling for all fitted effects (factors and covariates).
For essentially correlational analyses (principal components,
factor analysis, etc.),
similar displays can be used to show how multi-sample analyses
can be compromised by substantial group mean differences, and corrected
by analysis of the pooled within-sample covariance matrix, or by
including important group variables in the model.
Moreover, display of the the individual within-group data ellipses can
show visually how well the assumption of
equal covariance matrices,
$\Sigma_1 = \Sigma_2 = \dots = \Sigma_g$,
is satisfied in the data, for the two variables
displayed.

