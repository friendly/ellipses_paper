\subsection{Statistical ellipsoids}

In statistical applications, $\mat{C}$ will often be the inverse of a covariance
matrix (or a sum of squares and cross-products matrix), and the ellipsoid will
be centered at the means of variables, or at estimates of parameters under some model.
Hence, we will also use the following notation:

For a positive definite matrix
$\mat{\Sigma}$ we use $\mathcal{E}(\vec{\mu},\mat{\Sigma})$ to denote the ellipsoid
\begin{equation}\label{eq:ellipsoid3}
\mathcal{E} := \{ \vec{x} : (\vec{x}-\vec{\mu})\trans \inv{\Sigma} (x-\vec{\mu}) = 1 \} \period
 \end{equation}

When $\mat{\Sigma}$ is the covariance matrix of a multivariate vector $\vec{x}$ with eigenvalues
$\lambda_1 > \lambda_2 > \dots$,
the following
properties represent the ``size'' of the ellipsoid in $\Real{p}$:

\begin{tabular}{llll}
    Size                   &  Conceptual formula                    & Geometry       & Function \\
\hline
(a) Generalized variance:  & $\det{\mat{\Sigma}} = \prod_i \lambda_i$ & area, (hyper)volume & geometric mean\\  
(b) Average variance:        & $\trace{\mat{\Sigma}} = \sum_i \lambda_i $ & linear sum & arithmetic mean\\     %% superscripts turned to subscripts / GM
(c) Average variance:        & $1/ \trace{\mat{\Sigma}^{-1}} = 1/\sum_i (1/\lambda_i) $ &  & harmonic mean\\
(d) Maximal variance:      & $\lambda_1$ & maximum dimension & supremum
 \end{tabular}
\medskip

\noindent In multivariate tests, these correspond (with suitable transformations) to (a) Wilks's $\Lambda$,
(b) the Hotelling-Lawley  trace, (c) the Pillai trace, and (d) Roy's maximum root tests, as we describe
below in \secref{sec:mlm}.

Note that every non-negative definite matrix $\mat{W}$ can be factored as $\mat{W}=\mat{A}\mat{A}\trans$,
and the matrix $\mat{A}$ can always be selected so that it is square.
$\mat{A}$ will be non-singular if and only if $\mat{W}$ is non-singular.
A computational
definition of an ellipsoid that can be used for all non-negative definite matrices and that corresponds to the previous definition in the case of positive-definite matrices is
\begin{equation}
\mathcal{E}(\vec{\mu},\mat{W}) = \vec{\mu} + \mat{A} \mathcal{S} \comma
\end{equation}
where $\mathcal{S}$ is a unit sphere of conformable dimension and $\vec{\mu}$ is the centroid of the ellipsoid.
One convenient choice of \mat{A} is the Choleski square root, $\mat{W}^{1/2}$, as we describe in \secref{sec:conjugate}.
Thus, for some results below, a convenient notation in terms of $\mat{W}$ is
\begin{equation}\label{eq:ellipsoidW}
\mathcal{E}(\vec{\mu}, \mat{W}) = \vec{\mu} \oplus \sqrt{\mat{W}} = \vec{\mu} \oplus \mat{W}^{1/2} \comma
\end{equation}
where $\oplus$
emphasizes that the ellipsoid is a scaling and rotation of the unit sphere followed by translation to
a center at \vec{\mu} and $\sqrt{\mat{W}}=\mat{W}^{1/2}=\mat{A}$. This representation is not unique,
however:  $\vec{\mu} \oplus \mat{B} = \vec{\nu} \oplus \mat{C}$ (i.e., they generate the same ellipsoid)
\emph{iff} $\vec{\mu} = \vec{\nu}$ and $\mat{B}\mat{B}\trans = \mat{C}\mat{C}\trans$.
From this result, it is readily seen that under a linear transformation given by a matrix
$\mat{L}$
the image of the ellipse is

\begin{equation*}
\mat{L}(\mathcal{E}(\vec{\mu} ,\mat{W}))=\mathcal{E}(\mat{L}\vec{\mu} ,\mat{L}\mat{W}{\mat{L}}\trans)=\mat{L}\vec{\mu} \oplus \sqrt{\mat{L}\mat{W}{\mat{L}}\trans}=\mat{L}\vec{\mu} \oplus \mat{L}\sqrt{\mat{W}} \period
\end{equation*}

