\section{Introduction}

\epigraph{Whatever relates to extent and quantity may be represented by
geometrical figures. Statistical projections which speak to the senses without
fatiguing the mind, possess the advantage of fixing the attention on a great
number of important facts.}{Alexander von Humboldt \citeyearpar[p.~ciii]{Humboldt:1811a}}

In the beginning (of modern  statistical methods), there was the  ellipse. As statistical
methods progressed from bivariate to multivariate, the ellipse escaped the plane to a 3D
ellipsoid, and then onwards to higher dimensions.
This  paper extols and illustrates the
virtues of the ellipse and her higher-dimensional cousins for both didactic and
data analytic purposes.

When
Francis Galton  \citeyearpar{Galton:1886} first  studied the  relationship between  heritable traits of
parents and their offspring, he  had a remarkable visual insight--- contours of
equal bivariate frequencies in the joint distribution seemed to form  concentric
shapes whose outlines  were, to Galton,  tolerably close to  concentric ellipses
differing only in scale.

Galton's goal was to  to predict (or explain)  how a characteristic, $Y$,   (e.g.,
height) of children was  related to that of  their parents, $X$.  To  this end, he
calculated summaries,  $\mathrm{Ave}(Y\given X)$,  and, for  symmetry, $\mathrm{Ave}(X\given Y)$, and plotted
these as lines of means on his  diagram.  Lo and behold, he had a  second visual
insight:  the lines  of means of  ($Y\given X$) and ($X\given Y$)  corresponded approximately to
the locus of  horizontal and vertical  tangents to the  concentric ellipses.  To
complete the picture,  he added lines  showing the major  and minor axes  of the
family of ellipses, with the result shown in Figure 1.

It is  not stretching  the point  too far  to say  that a  large part  of modern
statistical  methods  descend  from  these  visual  insights:%
\footnote{\citet[p. 37]{Pearson:1920} later stated, ``that Galton
should have evolved all this from his observations is to my mind one
of the most noteworthy scientific discoveries arising from pure
analysis of observations.'' }
correlation   and
regression \citep{Pearson:1896}, the  bivariate normal  distribution,
and principal components  \citep{Pearson:1901,Hotelling:1933}  all trace their ancestry to Galton's  geometrical
diagram.%
\footnote{
Well, not entirely. Auguste Bravais [1811--1863] \citeyearpar{Bravais:1846}, an astronomer
and physicist first introduced the mathematical theory of the bivariate normal distribution
as a model for the joint frequency of errors in the geometric position of a point.
Bravais derived the formula for level slices as concentric ellipses and had a rudimentary
notion of correlation but did not appreciate this as a representation of data.
Nonetheless, \cite{Pearson:1920} acknowledged Bravais's contribution, and the correlation
coefficient is often called the Bravais-Pearson coefficient in France
\citep{Denis:2001}. }
%\todo{Ref: Denis}


Basic geometry goes back at least to Euclid, but the properties of the ellipse and  other
conic sections may be traced to Apollonius  of Perga (ca.~262 BC--ca.~190 BC),  a
Greek geometer and astronomer who gave the ellipse, parabola, and hyperbola their
modern names. In a work popularly called the Conics \citep{Boyer:91}, he  described
the  fundamental  properties  of  ellipses  (eccentricity,  axes,  principles of
tangency,  normals as  minimum and  maximum straight  lines to  the curve)  with
remarkable clarity nearly 2000 years before the development of analytic geometry
by Descartes.

% one figure
\begin{figure}[htb]
  \centering
  \includegraphics[width=.65\textwidth]{fig/galton-corr}
  \caption{Galton's 1886 diagram, showing the relationship of height of children
to the average of their parents' height. The diagram is essentially an overlay
of a geometrical interpretation on a bivariate grouped frequency distribution, shown
as numbers.}%
  \label{fig:galton-corr}
\end{figure}

Over time, the ellipse would be called to duty to provide simple explanations of
phenomena  once thought  complex.  Most  notable is  Kepler's insight  that the
Copernican theory of the orbits of planets as concentric circles (which required
notions  of  epicycles  to  account  for  observations)  could  be  brought into
alignment with the detailed observations by  Tycho Brahe and others by a  simple
law: ``The orbit  of every planet  is an ellipse  with the sun  at a focus.''  One
century later, Isaac  Newton was able  to derive all  three of Kepler's  laws as
simpler consequences of general laws of motion and universal gravitation.

This paper  takes up  the cause  of the  ellipse as  a geometric  form that  can
provide similar service to statistical understanding and data analysis.  Indeed,
it has been doing that since the time of Galton, but these graphic and geometric
contributions have often  been incidental and  scattered in the  literature
(e.g., \citet{Bryant:1984,CampbellAtchley:81,SavilleWood:1986,Wickens:1995}). 
We
focus here on visual  insights through ellipses in  the areas of linear  models,
multivariate linear models, and mixed-effect models.


\begin{comment}
%% two figures side-by-side
\begin{figure}[htb]
 \begin{minipage}[b]{.49\linewidth}
  \centering
  \includegraphics[width=1\linewidth]{fig/}
  \caption{}%
  \label{fig:}
 \end{minipage}%
 \hfill
 \begin{minipage}[b]{.49\linewidth}
  \centering
  \includegraphics[width=1\linewidth]{fig/}
  \caption{}
  \label{fig:}
 \end{minipage}
\end{figure}
\end{comment}
