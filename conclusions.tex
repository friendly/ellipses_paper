\section{Discussion and Conclusions}

\epigraph{I know of scarcely anything so apt to impress the imagination as the
wonderful form of cosmic order expressed by the ``(Elliptical) Law of Frequency of
Error.'' 
The law would have been personified by the Greeks and deified, if they had known of it. 
... % It reigns with serenity and in complete self-effacement, amidst the wildest confusion. 
    % The huger the mob, and the greater the apparent anarchy, the more perfect is its sway. 
    % It is the supreme law of Unreason. 
	%Whenever a large sample of chaotic elements are taken in hand 
	%and marshaled in the order of their magnitude, 
	%an unsuspected and most beautiful form of regularity proves to have been
	%latent all along.
}
{Sir Francis Galton, \emph{Natural Inheritance}, London: Macmillan, 1889.
(``(Elliptical)'' added).
%Quoted in J. R. Newman (ed.) The World of Mathematics, New York: Simon and Schuster, 1956. p. 1482.
}

In statistical data, theory and graphical methods, one main organizing
distinction can be made in \emph{all} 
of these depending
on the dimensionality of the problem.  A coarse but useful scale considers the essential defining
distinctions to be among:
\begin{itemize*}
 \item ONE (univariate), 
 \item TWO (bivariate), 
 \item MANY (multivariate).  
\end{itemize*}
This scale% 
\footnote{
This idea, as a unifying classification principle for data analysis and graphics
was first suggested to the first author in
seminars by John Hartigan at Princeton, c. 1968.
}
at least implicitly organizes much of current statistical teaching, practice, and software.
But within this, the data, theory and graphical methods are often treated separately (1D, 2D, $n$D),
without regard to geometric ideas and visualizations that help tie them together.

This paper starts from the idea that one geometric form--- the ellipsoid---
provides a unifying framework for many statistical phenomena, with simple representations in
1D (a line), 2D (ellipse) that extend naturally to $n$ dimensions.  The intellectual leap 
in statistical thinking from ONE to TWO in \citet{Galton:1886} was enormous.
Galton's visual insights from the ellipse quickly led to an understanding of the
ellipse as a contour of a bivariate normal surface.  From here, the step from TWO to MANY
would take another 20--30 years, but it is hard to escape the conclusion that 
geometric insight from the ellipse to the general ellipsoid in $n$D
played an important role in the development of multivariate statistical methods.

In this paper, we have tried to show how ellipsoids can be useful tools for
visual thinking, data analysis and pedagogy in a variety of contexts often
treated separately and from a univariate perspective.  Even in bivariate 
and multivariate problems, first-moment summaries (a 1D regression line
or 2+D regression surface) show only part of the story--- that of the
expectation of a response $\vec{y}$ given predictors $\mat{X}$.
In many cases, the more interesting part of the story concerns the 
\emph{precision} of various methods of estimation, which we've shown
to be easily revealed through data ellipsoids and
elliptical confidence regions for parameters.

The general relations among statistical methods, matrix algebra and geometry are
not new here.  To our knowledge, \citet{Dempster:69} was the first to exploit this
in a systematic fashion, establishing the connections among abstract vector spaces,
algebraic coordinate systems, matrix operations and properties, the dualities
between observation space and variable space,
 and the geometry
of ellipses and projections.%
%\footnote{
%The role of statistical computation should not be underestimated
%in appreciation of Dempster's contribution.
%...
%}
The roots of these connections go back much further---
to 
\citet{Cramer:1946} (idea of the concentration ellipsoid),
\citet{Hotelling:1933} (principal components)
and, we maintain, ultimately to \citet{Galton:1886}.
Throughout this development, elliptic geometry has played 
a fundamental role, leading to important visual insights.

The separate and joint roles of statistical computation and computational graphics should not be underestimated
in appreciation of these developments.  Dempster's analysis of the connections among geometry, algebra and
statistical methods was fueled by the development and software implementation of algorithms 
(Gram-Schmidt orthogonalization, Cholesky decomposition, sweep and multistandardize operators from
\citet{Beaton:64})
that allowed him to show precisely the translation of theoretical relations
from abstract algebra to numbers and thence to graphs and diagrams.  \citet{Monette:90}
took this several steps further, ...
 

Several features of the current discussion may help to present these ideas in a
new light.  

... [ to be completed ] ...




