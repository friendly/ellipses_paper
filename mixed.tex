\subsection{Mixed models: BLUEs and BLUPs}

In this section we make implicit use of the duality between data space and $\beta$
space, where lines in one map into points in the other and ellipsoids help to
visualize the precision of estimates in the context of the linear mixed
model for hierarchical data.  We also show visually how the best linear unbiased
predictors (BLUPs) from the mixed model can be seen as a weighted average
of the best linear unbiased estimates (BLUEs) derived from OLS regressions performed \emph{within} clusters of related data
and pooled OLS estimates computed \emph{ignoring} clusters.

The mixed model for hierarchical data provides a general framework for
dealing with dependence among observations in linear models,
such as occurs when students are sampled within schools, schools within
counties, and so forth
(e.g., \citealp{RaudenbushBryk:2002}). 
In these situations, the assumption of OLS that
the errors are conditionally independent is probably violated,
because, for example, students nested within the same school are
likely to have more similar outcomes than those from different schools. Essentially the same model, with provision for serially correlated errors, can be applied to longitudinal data (e.g., \citealp{LairdWare:1982}), although we will not pursue this application here.


The mixed model for the $n_i \times 1$ response vector $\vec{y}_i$ in
cluster $i$ can be given as
% \begin{equation}\label{eq:mixed0}
%  \vec{y}_i = \mat{X}_i \vec{\beta} + \mat{Z}_i \vec{u}_i + \vec{\epsilon}_i
% \end{equation}
\begin{eqnarray}
 \vec{y}_i & = & \mat{X}_i \vec{\beta} + \mat{Z}_i \vec{u}_i + \vec{\epsilon}_i \label{eq:mixed1} \\
 \vec{u}_i & \sim & \mathcal{N}_q (\vec{0}, \mat{G_i}) \nonumber\\
 \vec{\epsilon}_i & \sim & \mathcal{N}_{n_i} (\vec{0}, \mat{R}_i) \nonumber
\end{eqnarray}
where
%$\vec{y}_i$ is the $n_i \times 1$ response vector of observations in the $i$th stratum,
$\vec{\beta}$ is a $p \times 1$ vector of parameters corresponding to the fixed effects
in the $n_i \times p$ model matrix $\mat{X}_i$;
$\vec{u}_i$ is a $q \times 1$ vector of coefficients corresponding to the random effects
in the $n_i \times q$ model matrix $\mat{Z}_i$;
$\mat{G_i}$ is the $q \times q$ covariance matrix of the random effects in $\vec{u}_i$;
and $\mat{R}_i$ is the $n_i \times n_i$ covariance matrix of the errors in $\vec{\epsilon}_i$.

Stacking the $\vec{y}_i$, $\mat{X}_i$, $\mat{Z}_i$, and so forth in the obvious way then gives
\begin{equation}\label{eq:mixed2}
 \vec{y} = \mat{X} \vec{\beta} + \mat{Z} \vec{u} + \vec{\epsilon}
\end{equation}
where $\vec{u}$ and $\vec{\epsilon}$ are assumed to have normal distributions with mean $\vec{0}$
and
\begin{equation}\label{eq:mixed3}
 \Var
      \begin{pmatrix}
       \vec{u} \\ \vec{\epsilon}
      \end{pmatrix}
      =
      \begin{bmatrix}
      \mat{G} & \mat{0} \\ \mat{0} & \mat{R}
      \end{bmatrix} \comma
\end{equation}
where $ \mat{G} = \diag(\mat{G}_1,...,\mat{G}_m)$, $\mat{R} = \diag(\mat{R}_1,...,\mat{R}_m)$ and $m$ is the number of clusters.  
The variance of $\vec{y}$ is therefore $\mat{V} = \mat{Z} \mat{G} \mat{Z}\trans + \mat{R}$, and when
$\mat{Z} = \mat{0}$ and $\mat{R} = \sigma^2 \mat{I}$, the mixed model in \eqref{eq:mixed2} reduces to the
standard linear model.

%%%%%%%%%%%%%%%%
% changes from Georges

We now consider the case in which 
$\mat{Z}_i = \mat{X}_i$ and we wish to predict $\vec{\beta}_i = \vec{\beta} + \vec{u}_i$, the vector of parameters for the $i_{th}$ cluster. 
At one extreme, we could simply ignore clusters and use the common mixed-model generalized-least-square estimate,
\begin{equation}\label{eq:mixed4}
 \widehat{\vec{\beta}}^{\textrm{GLS}} = (\mat{X}\trans \mat{V}^{-1} \mat{X})^{-1} \mat{X}\trans \vec{y}
\end{equation}
whose sampling variance is $\Var(\widehat{\vec{\beta}}^{\textrm{GLS}}) = \inv{(\mat{X}\trans \mat{V}^{-1} \mat{X})}$. 
It is an unbiased predictor of $\vec{\beta}_i$ since $E( \widehat{\vec{\beta}}^{\textrm{GLS}} - \vec{\beta}_i) = 0$. 
With moderately large $m$, the sampling variance may be small relative to $\mat{G}_i$ and 
$\Var(\widehat{\vec{\beta}}^{\textrm{GLS}} - \vec{\beta}_i) \approx \mat{G}_i$.

At the other extreme, we ignore the fact that clusters come from a common population and we calculate the separate 
BLUE estimate within each cluster,
\begin{equation}\label{eq:mixed5}
 \widehat{\vec{\beta}}^{\textrm{blue}}_i = \inv{(\mat{X}_i\trans \mat{X}_i)} \mat{X}_i\trans \vec{y}_i
 \quad\quad \textrm{with} \quad\quad
 \Var(\widehat{\vec{\beta}}^{\textrm{blue}}_i \given \vec{\beta}_i) \equiv \mat{S}_i = 
 \hat{\sigma}^2 \inv{(\mat{X}_i\trans \mat{X}_i)} \period
\end{equation}

Both extremes have drawbacks: whereas the pooled overall GLS estimate ignores variation between clusters, 
the unpooled within-cluster BLUE ignores the common population and makes clusters appear to differ more than they actually do.


This dilemma led to the development of BLUPs (best linear unbiased predictor) in models with random effects \citep{Henderson:1975,Robinson:1991,Speed:1991}.
In the case considered here, the BLUPs are an inverse-variance weighted average of the mixed-model GLS estimates and of the BLUEs. The BLUP is then:

% ::$ \tilde{\beta}_i^{blup} = (S_i^{-1} + G_i^{-1})^{-1} (S_i^{-1}\hat{\beta}_i^{blue} + G_i^{-1}\hat{\beta}^{gls})$       (34)

\begin{equation}\label{eq:mixed6}
 \widetilde{\vec{\beta}}^{\textrm{blup}}_i =
 \inv{ \left( \inv{\mat{S}_i} + \inv{\mat{G}_i} \right) } 
 \left(
   \inv{\mat{S}}_i \widehat{\vec{\beta}}^{\textrm{blue}}_i + 
   \inv{\mat{G}}_i \widehat{\vec{\beta}}^{\textrm{GLS}}_i
 \right)
\period
\end{equation}
This "partial pooling" optimally combines the information from cluster $i$ with the information from all clusters, 
shrinking $\widehat{\beta}_i^{\textrm{blue}}$ towards $\widehat{\beta}^{\textrm{GLS}}$. 
Shrinkage for a given parameter $ \beta_{ij} $ is greater when the sample size $n_i$ is small or when the variance of the 
corresponding random effect, $g_{ijj}$, is small. 



%%%%%%%%%%%%%%%%%
\begin{comment}
Assume that our interest lies primarily in estimating the fixed-effect parameters in $\vec{\beta}$.  At one extreme
(complete pooling) we could simply ignore clusters and calculate the pooled OLS estimate,
\begin{equation}\label{eq:mixed4}
 \widehat{\vec{\beta}}^{\textrm{pooled}} = \inv{(\mat{X}\trans \mat{X})} \mat{X}\trans \vec{y}
 \quad\quad \textrm{with} \quad\quad
 \widehat{\mat{S}} \equiv \widehat{\Var}(\widehat{\vec{\beta}}^{\textrm{pooled}}) = \hat{\sigma}^2 \inv{(\mat{X}\trans \mat{X})} \period
\end{equation}
At the other extreme (no pooling), we can ignore the variation across clusters and calculate the
separate BLUE estimate within each cluster, giving the results of a fixed-effects analysis,
\begin{equation}\label{eq:mixed5}
 \widehat{\vec{\beta}}^{\textrm{unpooled}}_i = \inv{(\mat{X}_i\trans \mat{X}_i)} \mat{X}_i\trans \vec{y}_i
 \quad\quad \textrm{with} \quad\quad
 \widehat{\mat{S}}_i \equiv \widehat{\Var}(\widehat{\vec{\beta}}^{\textrm{unpooled}}_i) = \hat{\sigma}^2 \inv{(\mat{X}_i\trans \mat{X}_i)} \period
\end{equation}
Both extremes have drawbacks:  whereas the pooled analysis ignores variation among clusters, the unpooled analysis ignores the
cluster-averaged result and overstates variation within each cluster, making the clusters appear to differ more than they actually do.

This dilemma led to the development of BLUPs in models with random effects \citep{Henderson:1975,Robinson:1991,Speed:1991}.
In the case considered here, $\mat{Z}_i = \mat{X}_i$, and so $\widehat{\vec{u}}_i$ gives ``estimates'' of the random effects with
$\widehat{\Var} (\widehat{\vec{u}}_i) = \widehat{\mat{G}}$. The BLUPs are a weighted average of the $\widehat{\vec{\beta}}_i$
and $\widehat{\vec{u}}_i$ using the precision ($\Var^{-1}$) as weights,

\begin{equation}\label{eq:mixed6}
 \widetilde{\vec{\beta}}^{\textrm{blup}}_i =
 \left(
   \widehat{\vec{\beta}}^{\textrm{unpooled}}_i \inv{\widehat{\mat{S}}_i}  + \inv{\widehat{\vec{u}}_i \widehat{\mat{G}}}
 \right)
 \inv{ \left( \inv{\widehat{\mat{S}}_i} + \inv{\widehat{\mat{G}}} \right) } \period
\end{equation}
This ``partial pooling'' optimally combines the information from cluster $i$ with the information from all clusters,
shrinking the $\widehat{\vec{\beta}}_i$ toward $\widehat{\vec{\beta}}^{\textrm{pooled}}$. Shrinkage
for a given parameter $\beta_{ij}$ is greater
when the sample size $n_i$ is small or when the estimated variance of the corresponding 
random effect, $g_{jj}$, is small.
\end{comment}


\eqref{eq:mixed6}
is of the same form as \eqref{eq:bayes-posterior} and other convex combinations of estimates considered
earlier in this section. So once again, we can understand these results geometrically as the locus of
osculation of ellipsoids. Ellipsoids kiss for a reason: to provide an optimal 
convex combination of information from two sources, taking precision into account.


