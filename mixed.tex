\subsection{Mixed models: BLUEs and BLUPs}
\TODO{This section, and the incomplete example that follows is just an initial attempt,
awaiting a more insightful description.}

In this section we make use of the duality between data space and $\beta$
space, where lines in one map into points in the other and ellipsoids to
visualize the precision of estimates in the context of the general linear mixed
model for hierarchical data.  We also show visually how the best linear unbiased
predictors (BLUPs) from the mixed model can be seen as a weighted average
of OLS regression, best linear unbiased estimates (BLUEs) \emph{within} strata
and the variation of random effect estimates \emph{between} strata.

The mixed model for hierarchical data provides a general framework for 
dealing with lack of independence among observations in linear models,
such as occurs when students are sampled within schools, schools within
counties and so forth. In these situations, the assumption of OLS that
the residuals are conditionally independent is likely to be violated,
because, for example, students nested within the same school are 
likely to have more similar outcomes than those from separate schools.

